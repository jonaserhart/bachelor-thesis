% Chapter Template

\chapter{Conclusion} 

\label{Chapter7} 

\section{Summary}

\section{Contribution}

It can be argued that this thesis contributes to the world of SCRUM and KPIs in two different ways: It extends the knowledge base about the demand for and usage of KPIs and contributes more practically to the actual evaluation and presentation of KPIs. These contributions will be further discussed in this section. 

\subsection{KPI usage and demand}

This thesis is yet another testimonial to how essential it is to use indicators to improve performance. Surveys, statistics, and the survey done in the context of this thesis all point to the fact that the people working with SCRUM find KPIs helpful for their process. These demands for KPIs differ vastly according to the team that answers these questions. It should be noted that some KPIs are likely to be more helpful and have a broader application for more teams. But it is also important to see that there is a demand by SCRUM actors for having custom KPIs to better analyze the performance of their specific team. All in all, this thesis helps define KPI usage and demands a little further.

\subsection{KPI evaluation}

The artifact produced during this thesis can ease the computation of complex KPIs by providing a user interface to do so. Teams can use this application to create templates for sprint reports. They can create custom KPIs and data sources to acquire data about their SCRUM process. To present the gathered data to their team members, custom dashboards can be created with ease. All these capabilities make it easier for teams to assess their performance and improve using custom metrics.


\section{Reflection on the process}