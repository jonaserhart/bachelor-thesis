% Chapter Template

\chapter{Conclusion} 
\label{Chapter7} 

This chapter concludes this thesis and includes my reflection on the whole process.

\section{Summary}

This thesis included a short exploration of KPIs, their helpfulness and their application.
It has been stated multiple times across the last fifty pages that there is no finite
set of KPIs that satisfies every team's requirements when it comes to evaluating and improving SCRUM processes.
This created a need for a solution.
A first attempt at the solution is presented in this thesis in the form of an artifact designed with the DSR framework.
The application's design was described in detail and it was evaluated by the partnering company VertiGIS.
The requirements agreed upon in the beginning were satisfied and the application will be developed further to better cater to the team's needs.
The application was developed to be as widely usable as possible, providing support for multiple SCRUM logging backends.


\section{Contribution}

It can be argued that this thesis contributes to the world of SCRUM and KPIs in two different ways: 
It extends the knowledge base about the demand for and usage of KPIs and contributes more practically to the actual evaluation and presentation of KPIs. 
These contributions will be further discussed in the following sections. 

\subsection{KPI usage and demand}

This thesis is yet another testimonial to how essential it is to use indicators to improve performance. 
Surveys, statistics, and the survey done in the context of this thesis all point to the fact that the people working with SCRUM find KPIs helpful for their process. 

These demands for KPIs differ vastly according to the team that answers these questions. 
It should be noted that some KPIs are likely to be more helpful and have a broader application for more teams. 
But it is also important to see that there is a demand by SCRUM actors for having custom KPIs to better analyze the performance of their specific team. 

All in all, this thesis helps define KPI usage and demands a little further.

\subsection{KPI evaluation}

The artifact produced during this thesis can ease the computation of complex KPIs by providing a user interface to do so. 
Teams can use this application to create templates for sprint reports. 
They can create custom KPIs and data sources to acquire data about their SCRUM process. 

To present the gathered data to their team members, custom dashboards can be created with ease. 
All these capabilities make it easier for teams to assess their performance and improve using custom metrics.


\section{Reflection on the process}

This section will feature my reflection on the whole process of writing the thesis and the artifact.

\subsection{Lessions learned}

\subsubsection*{DSR - a helpful tool}

At first, the DSR framework might be a little bit difficult to implement or comprehend.
But once I understood the framework it was a powerful tool to follow and use as a guideline for a scientific project. 

\subsubsection*{Test-driven development}

As a fan of test-driven development, it's hard to admit that it hindered me in the first stages of the artifact development.
When faced with limited time on a small project, test-driven development includes having an overhead that is comparable with the implementation itself.
Unit tests were written for finished parts to find bugs and errors after the code was written and were very helpful, but the attempt to develop this whole artifact test-driven was dropped a few weeks into development.

\subsubsection*{Survey evaluation}

Having limited practical experience with data analysis, I struggled to evaluate survey data to its full extent at first.
After doing some research and visiting some older university course material, I understood exactly how I could evaluate, categorize and correlate
the data from my surveys with programming tools like Python. The machine learning course I finished halfway into my thesis contributed a lot to my knowledge of data analysis.


\subsection{Reflection}

I estimated the time it took me to write this thesis to be about 7 months which was almost correct. 
This was a surprise because I had never written a thesis of this length before. 
The practical part of building the application went very well and I am proud to say that the code 
is one of the cleanest pieces I have ever written as a software developer that has been employed for about 4 and a half years now.
The papers I have read during the research for this paper were very interesting, especially the master's thesis of Alexandra Esguerra \parencite{AgileProjectHealthIndicatorsThesis}.
I predicted that the application would need more development outside the context of this thesis for it to be used widely, which is the case.
I am happy to develop this artifact further and provide it to more teams who want to improve their SCRUM retrospective.
