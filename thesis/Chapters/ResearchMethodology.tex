\chapter{Research Methodology}

The following sections will describe how this project uses 
Design Science Research to ensure that the work is both practically 
relevant and academically sound.

\label{Chapter3}
\section{Overview of Design Science Research}

\enquote{
Simply stated, Design Science Research seeks to enhance
technology and science knowledge bases via the creation of innovative artifacts that solve
problems and improve the environment in which they are instantiated. The results of DSR include
both the newly designed artifacts and design knowledge (DK) that provide a fuller understanding
via design theories of why the artifacts enhance (or, disrupt) the relevant application contexts.
}  

  
- \cite{DesignScienceHevner}

\subsection{Characteristics of DSR}

DSR is a well-defined framework characterized by the following points and principles:

\begin{enumerate}
    \item \textbf{Artifact-Centric}: The core of DSR is the creation of artifacts. These can range from practical software applications and hardware systems to theoretical models and methods.
    \item \textbf{Problem-Solving}: DSR is inherently problem-oriented. It aims to address real-world challenges by developing practical solutions.
    \item \textbf{Iterative Development}: The methodology often involves an iterative process of design, development, and refinement. This allows for the continuous improvement of the artifact.
    \item \textbf{Evaluation}: A crucial aspect of DSR is the evaluation of the artifact to ensure it meets its intended objectives and solves the problem at hand.
    \item \textbf{Knowledge Contribution}: While the primary goal is problem-solving, DSR also aims to contribute to the body of knowledge by documenting the design process, the artifact, and the evaluation results.
\end{enumerate}

\subsection{Relevance to Research}

DSR is particularly relevant in fields like Information Systems, 
Computer Science, and Engineering, 
where the focus is not just on understanding problems 
but also on creating solutions. 
It provides a structured framework for the development 
and evaluation of artifacts, ensuring both practical 
utility and academic rigor.

\subsection{Methodological Steps}

Typically, a DSR project involves the following steps:

\begin{enumerate}
    \item \textbf{Problem Identification}: Understand and define the problem that needs solving.
    \item \textbf{Objective Setting}: Establish what the artifact aims to achieve.
    \item \textbf{Design \& Development}: Create the artifact based on the objectives.
    \item \textbf{Demonstration}: Show that the artifact works in a real-world scenario.
    \item \textbf{Evaluation}: Assess the artifact's effectiveness and efficiency.
    \item \textbf{Communication}: Document the results.
\end{enumerate}

The following sections will discuss how DSR will be used for this project and thesis
specifically.

\section{Artifact design and construction}
Artifact design in DSR heavily depends on the identification of the problem.
In the case of this thesis, the problem is described in Chapter \ref{Chapter1}.
After identifying the problem an objective was defined in 
section \ref{Chapter1-Objectives}. 
Based on this the application will be developed and evaluated. 
This thesis includes diagrams and possible scenarios the application should be able to handle in Chapter \ref{Chapter5}. 
These were created before the actual implementation as a guide
and then updated during the first development cycle. 
The construction of the artifact started with setting up a new software project and implementing the planned data models,
services, and APIs according to the specification created in the planning stage. 

\section{Artifact evaluation}

The most important factors for evaluating this artifact 
are the time a developer has to spend on creating a sprint
analysis and the completeness of the KPI spectrum. These factors are
determined by the partnering company to match the current method of evaluating
team performance.
Additionally, a 'good' artifact will also be defined by its usability, 
the easy process of deployment, and the lack of unexpected behavior. 
The definition of functional requirements and non-functional requirements will 
be helpful in determining these factors. 
More on evaluation in Chapter \ref{Chapter6}.
